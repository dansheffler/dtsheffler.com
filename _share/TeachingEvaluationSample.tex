%!TEX TS-program = xelatex
%!TEX encoding = UTF-8 Unicode

\RequirePackage{etoolbox}
            \documentclass[12pt,oneside]{book}
    

\usepackage{amsmath,amssymb}

\usepackage{ctable}
\setupctable{width=2in}
\usepackage{placeins}


% Diagrams
\usepackage{tikz}
\usetikzlibrary{arrows}
\usepackage{subcaption}


% Use fonts from system fonts.
\usepackage{fontspec,xunicode}
\defaultfontfeatures{Mapping=tex-text}
\setmainfont[Mapping=tex-text]{Palatino Linotype}
%\settitlefont[Mapping=tex-text]{OFL Sorts Mill Goudy}
%\setsansfont[Scale=MatchLowercase,Mapping=tex-text]{Helvetica Neue}
%\setmonofont[Scale=MatchLowercase]{Consolas}
\newfontfamily\greekfont[Scale=1]{GFS Porson}

% Ostensibly Spaces will typeset in mathmode.
% Doesn't seem to work, us \textrm{} instead.
{\catcode`\ =\active{\global\let =\ }}
\everymath{\catcode`\ =\active}



% Get \boxright from txfonts without changing math font face.
% \boxright can only be used in mathmode.
\DeclareSymbolFont{symbolsC}{U}{txsyc}{m}{n}
\DeclareMathSymbol{\boxright}{\mathrel}{symbolsC}{128}


%%%%%%%%%%%%%%%%%%%%%%%%%%%%%%%%
% Get Greek to work automatically.
% Just type Greek unicode in the body of the text and this will detect when the switch has been made.

\usepackage{polyglossia}
\setdefaultlanguage{english}
\setotherlanguage[variant=ancient]{greek}


\input{C:/Users/Dan/texmf/tex/greektokens}

\XeTeXinterchartoks 0 \greekLetter = {\begin{otherlanguage*}{greek}}
\XeTeXinterchartoks \greekLetter 0 = {\end{otherlanguage*}}
\XeTeXinterchartoks 255 \greekLetter = {\begin{otherlanguage*}{greek}}
\XeTeXinterchartoks \greekLetter 255 = {\end{otherlanguage*}}

%%%%%%%%%%%%%%%%%%%%%%%%%%%%%%%%%%%%%%%%%


% Pandoc's Meta-data handling

% \newcommand{\mytitle}{testing}
% \newcommand{\Originaltitle}{}

% \let\Originaltitle\title
% \renewcommand*{\title}[1]{%
%   \Originaltitle{#1}%
%   \renewcommand*{\mytitle}{#1}%
% }

% \newcommand{\myauthor}{testing}
% \newcommand{\Originalauthor}{}

% \let\Originalauthor\author
% \renewcommand*{\author}[1]{%
%   \Originalauthor{#1}%
%   \renewcommand*{\myauthor}{#1}%
% }

% \newcommand{\mydate}{testing}
% \newcommand{\Originaldate}{}

% \let\Originaldate\date
% \renewcommand*{\date}[1]{%
%   \Originaldate{#1}%
%   \renewcommand*{\mydate}{#1}%
% }






% % Give some default values.
% \def\myauthor{}
% \def\mydate{}
% \def\teacher{}
% \def\class{}
% \def\semester{}
% \def\dearline{To whom it may concern,}


%Pandoc subscript command
\newcommand{\textsubscr}[1]{\textsubscript{#1}}


% My finicky block quote formating
\usepackage[leftmargin=0.5in,rightmargin=0.5in,noorphans,indentfirst=false]{quoting}

% Special No Indent Command
\makeatletter
\newcommand*{\@doendeq}{%
  \everypar{{\setbox\z@\lastbox}\everypar{}}%
}

\renewenvironment{quote}{%
    \singlespacing\begin{quoting}%
}{%
    \end{quoting}\ignorespacesafterend\par\noindent\aftergroup\@doendeq%
}

\makeatother




% Make Footnote fontsize the same as the body text.
\let\origfootnotesize\footnotesize
\let\footnotesize\normalsize


% Chicago style footnotes (full citation in first note)
\usepackage[notes,isbn=false,url=false]{biblatex-chicago}
\addbibresource{mybib.bib}


% Fix footnote number indentation
\usepackage{footmisc}
\makeatletter
\renewcommand\@makefntext[1]{% 
    \parindent 0.5in% 
    \@thefnmark.~#1}
\makeatother


%Automatic Links
\usepackage[xetex, bookmarks, colorlinks=false, pdftitle={Samples from Teaching Evaluations}, pdfauthor={Dan Sheffler}, pdfsubject={}]{hyperref}



% Pandoc's expected environment for lists
\usepackage[shortlabels]{enumitem}
\setlistdepth{20}

\usepackage{afterpage}


%%%%%%%%%%%%%%%%%% Pandoc's Presets %%%%%%%%%%%%%%%%%%%%%%%%%%%

% use upquote if available, for straight quotes in verbatim environments
\IfFileExists{upquote.sty}{\usepackage{upquote}}{}
% use microtype if available
\IfFileExists{microtype.sty}{%
\usepackage{microtype}
\UseMicrotypeSet[protrusion]{basicmath} % disable protrusion for tt fonts
}{}
\usepackage{longtable,booktabs}




%%%% Important Metadata %%%%
\title{Samples from Teaching Evaluations}
\author{Dan Sheffler}
\date{2016-02-23}




%%%%%%%%%% Chicago %%%%%%%%%%%%%





% Page layout
\usepackage[right=1in,left=1in,top=1in,bottom=1in,includehead]{geometry}
\usepackage{setspace}
\parindent  0.5in

% List Environments
\setlist[enumerate]{%
    labelindent=0.5in,%
    leftmargin=*,%
     nosep%
}
\setlist[itemize]{%
    leftmargin=0.5in,%
    labelindent=0.0in,%
    labelsep=*%
}


% Get single-spacing for paragraphs in numbered lists

\makeatletter
\let\oldenumerate=\enumerate
\renewenvironment{enumerate}{%
    \singlespacing\begin{oldenumerate}
}{%

    \end{oldenumerate}\ignorespacesafterend\par\noindent\aftergroup\@doendeq}
\makeatother



%Title, Section, etc. Subsubsections blank in the text body (paragraph labels), but the text *is* included in the TOC.

\usepackage[bottomtitles,explicit]{titlesec}
\titleformat{\chapter}[display]{\vspace{1in}\bfseries\Huge}{Chapter \thechapter \LARGE\newline #1}{0.2in}{}[\vspace{-1in}]
\titleformat{\section}[hang]{\bfseries\large}{\Roman{section}~~--~~#1}{0in}{}
\titleformat{\subsection}[hang]{\bfseries\normalsize}{#1}{0in}{}
\titleformat{name=\section,numberless}[hang]{\bfseries\large}{#1}{0in}{}
% \titleformat{\subsection}[runin]{}{}{.5in}{}
\titleformat{\subsubsection}[runin]{\normalsize}{}{0.5in}{\hspace{0.5in}}


\titlespacing*{\subsubsection}{0.5in}{0in}{0in}


% \makeatletter
% \newcommand*{\@doendeq}{%
%  \everypar{{\setbox\z@\lastbox}\everypar{}}%
%}


% Reset subsubsection counters even if levels are skipped.
\makeatletter
    \@addtoreset{subsubsection}{section}
\makeatother





% Page Header Formating
\usepackage{fancyhdr}
\setlength{\headheight}{15.2pt}
\usepackage{lastpage}


\fancypagestyle{plain}{%
\fancyhf{} % clear all header and footer fields
\fancyfoot[C]{\thepage} % except the center
\renewcommand{\headrulewidth}{0pt}%
\renewcommand{\footrulewidth}{0pt}%
\newgeometry{top=1in,left=1in,right=1in,bottom=1in}
}

\fancypagestyle{none}{%
\fancyhf{} % clear all header and footer fields
\renewcommand{\headrulewidth}{0pt}%
\renewcommand{\footrulewidth}{0pt}%
\newgeometry{top=1in,left=1in,right=1in,bottom=1in}
}

\fancypagestyle{myheading}{%
\lhead{\rightmark}\chead{}\rhead{Sheffler \thepage}
\fancyfoot[C]{}%
\renewcommand{\headrulewidth}{1pt}%
\renewcommand{\footrulewidth}{0pt}%
\restoregeometry%
}





\clubpenalty=1000
\widowpenalty=1000




%%%%%%%%%%%%%%%%%%%%%%%%%%%%%%%%%%%%%%%%%%%

\begin{document}

%Fixes for sections in the header.
\pagestyle{myheading}
\renewcommand{\chaptermark}[1]{\markright{\chaptername\ \thechapter\ -\ #1}}
\renewcommand{\sectionmark}[1]{\markright{\Roman{section}\ -\ #1}}


    % Exclude header from first page
    \thispagestyle{plain}

    % %Class information
    % \noindent Dan Sheffler\\
    % \noindent\LARGE\textbf{Samples from Teaching Evaluations}
    % \\
    % \\

    % % Title
    % \vspace{.1in}


    % The rest of the paper

\doublespace




%%%%%%% Pandoc Presets %%%%%%%

\begin{longtable}[c]{@{}lllll@{}}
\toprule
Semester & Course & Overall Quality testing testing testing testing
testing & Overall Value & Approx. Number\tabularnewline
\midrule
\endhead
2011 Spring & PHI 120-010 & 3.8 & 3.1 & 25\tabularnewline
2011 Spring & PHI 120-017 & 3.4 & 3.1 & 25\tabularnewline
2011 Spring & PHI 120-018 & 3.4 & 2.7 & 25\tabularnewline
2011 Fall & PHI 120-021 & 3.8 & 3.5 & 35\tabularnewline
2011 Fall & PHI 120-025 & 3.7 & 3.6 & 35\tabularnewline
2012 Spring & PHI 130-010 & 3.5 & 3.3 & 15\tabularnewline
2012 Fall & PHI 130-003 & 3.8 & 3.6 & 30\tabularnewline
2012 Fall & PHI 130-004 & 3.5 & 3.3 & 30\tabularnewline
2013 Fall & PHI 100-010 & 4.0 & 3.8 & 30\tabularnewline
2013 Fall & PHI 334-001 & 3.9 & 3.6 & 30\tabularnewline
2013 Spring & PHI 334-002 & 3.8 & 3.6 & 35\tabularnewline
2014 Fall & PHI 100-006 & 4.0 & 3.6 & 50\tabularnewline
2014 Fall & PHI 100-011 & 3.9 & 3.8 & 20\tabularnewline
\bottomrule
\end{longtable}

PHI 100 = Introduction to Philosophy PHI 120 = Introduction to Logic PHI
130 = Introduction to Ethics PHI 334 = Business Ethics

Overall quality of teaching is marked on a scale of 0 to 4.

\section{Introduction to Philosophy (PHI
100)}\label{introduction-to-philosophy-phi-100}

``{[}If you were teaching this course, what would you do differently?{]}
I'd resign and have Mr.~Sheffler teach it so students could have a
better education.''

\section{Introduction to Logic (PHI
120)}\label{introduction-to-logic-phi-120}

``I was never bored. Instead he woke me up and dragged me into
philosophy.''

``Dan was an excellent instructor and made some very complicated themes
fairly simple, he always made himself available for help outside of the
classroom, he will be a very good professor.''

``Strengths include the informality of the course. Instructor was not
intimidating or rude. Very helpful.''

``The instructor's strengths were his ability to gauge student interest,
answer questions effectively, and come across as caring about his
students.''

``Mr.~Sheffler covered all the material efficiently and in a relaxed,
easy manner that was conducive to my way of learning.''

``The class was structured very nicely because there was lecture and
discussion. Our professor called on students so everyone had to
participate.''

``{[}He is{]} very smart, very interesting/stimulating. Great teacher.
Fair grader, encourages success. No weaknesses.''

\section{Business Ethics (PHI 334)}\label{business-ethics-phi-334}

``I gained a great deal of valuable opinions and views as well as
thought provoking discussions. {[}The instructor's strengths are{]}
excellent knowledge, demeanor and preparedness.''

``I liked how he encouraged student participation and made everyone feel
comfortable to speak up.''

``{[}He is{]} really good at making class enjoyable and carrying out
discussions; has great knowledge of subject.''

``He encourages everyone to participate and voice their opinions. The
instructor is very knowledgeable about the material and presents it
well.''

\end{document}
